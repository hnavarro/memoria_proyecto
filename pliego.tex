\section{Consideraciones previas}
\label{cha:rendimiento}
\par En esta sección trataré de exponer la problemática asociada a la asignación razonada de requisitos.
Hemos dejado claro que nuestra apuesta será Drupal. En concreto, la versión de la rama 6.x, que está debidamente probada y se supone estable.
He buscado información sobre los requerimientos asociados a ejecutar Drupal y esta investigación creo que se podría dividir en dos partes, aunque están interconectadas y no deben ser aisladas del todo a la hora de tomar en consideración: las necesidades de rendimiento y las particularidades del software elegido.
\subsection{Rendimiento necesario}
\par ¿Qué quiero expresar con esto? Como es bien sabido en ingeniería, a menudo (por no decir constantemente), nos encontramos en disyuntivas que conducen al compromiso entre características y costes. Este caso no es distinto. En concreto, necesitamos definir qué necesidades tenemos y ver hasta dónde podemos llegar en materia de costes. 
\par En nuestro caso podríamos decir que tenemos, como referencia previa, la actual página dedicada a museo virtual en la página del IGN\cite{references:museovirtualoan}, pero, como ya hemos hablado anteriormente, existen numerosas diferencias entre los sitios que deberían ser consideradas, a saber: la casi nula visibilidad de la página actual, la inexistente publicidad de la misma, el acceso exclusivo desde internet y la no asociación a ningún fenómeno ``vivo''. 

\subsubsection{Afluencia}
\par En el caso del sitio que intentamos crear, tenemos un fenómeno asociado que es la apertura (por primera vez en su historia) del recinto del ROM al público general, hasta ahora, en forma de visitas guiadas que pasan de manera no exhaustiva sobre algunos de los objetos referidos en nuestro museo virtual. Al tener una gestión más cercana y ágil, la idea es que los dos fenómenos (visitas físicas y visitas virtuales) evolucionen a la vez apoyándose y refiriéndose el uno al otro. Dado este hecho, podemos suponer que la afluencia virtual será más alta que la actual (desconocida) aunque desgraciadamente es muy difícil tener una estimación decente sobre este aspecto. Si nos basamos en el cálculo de visitas físicas actual y suponemos un ``worst-case scenario'' en el que todos los grupos de visitas guiadas están llenos tendríamos 25 personas por grupo y 8 vistas por semana. Lo que nos daría 200 vistas seguras a la semana, suponiendo que todos los visitantes repitan virtualmente (cosa harto difícil dado el gran número de asistentes que pertenecen a grupos poco familiarizados con las nuevas tecnologías) y que lo hagan una vez (cosa que puede variar). Por esta métrica, tendríamos una afluencia baja, que podría variar a lo largo del tiempo pero no se estima que fuesen cambios muy bruscos o muy significativos.

\subsubsection{Caudal}
\par Para nuestro museo, si bien no tendremos una afluencia muy elevada, sí es verdad que serviremos, además de texto, contenido multimedia como imágenes (fundamentalmente), vídeos y audios, lo que podría suponer un tráfico más elevado y una carga más alta al servidor. Yo anotaría como posibles recursos a proteger un ancho de banda razonable y una capacidad de almacenamiento nada restringida (aunque sin llegar a cifras especiales). La potencia de CPU no veo que vaya a estar afectada mucho por estos fenómenos dada la naturaleza de los mismos si está bien implementada.

\subsubsection{Criticidad}
\par Añadido casi de manera anecdótica, podríamos decir que la criticidad de este sistema es muy baja, siendo, quizá, más importante la integridad y custodia de los datos que la disponibilidad total de los mismos. Es decir, si nuestro servidor se cae eventualmente o funciona más lento en picos de tráfico (cosa harto difícil) no supondrá un perjuicio apreciable para la empresa.

\subsection{Particularidades del software elegido}
\par Como toda solución tecnológica, la plataforma elegida (Drupal) tiene asociados unos requisitos, que veremos más adelante, y unos desempeños más o menos perfilados por los componentes que integra y/o necesita para funcionar.
En cuestiones de rendimiento hay mucho escrito sobre Drupal. Como no es el propósito de este trabajo analizar el rendimiento de manera escrupulosa (dado que estamos hablando de una máquina no crítica), voy a dar una visión general sobre el tema.
\par Para ello, tras hacer unas búsquedas, encontré, además de otras fuentes más específicas\cite{references:drupalperf2bits}\cite{references:drupalperferikwebb}, un artículo\cite{references:drupalservertuning} entre la documentación de Drupal disponible en su sitio web. En él se expone de manera clara y concisa la particularidad del sistema. A continuación, glosaré algunas de las ideas allí expuestas (original en inglés).
\par Además de los apartados anteriores, este artículo pone énfasis en la forma en la que está construida la pila de procesos de Drupal. Una buena consideración siempre que nos referimos a configuraciones hardware y software es la prudencia. Prudencia en la toma de decisiones y prudencia en los cambios.
\subsubsection{Rendimiento de la pila LAMP}
\par LAMP (Linux + Apache + MySQL + PHP) es la pila de procesos sobre la que correrá Drupal. Si bien puede cambiarse, por ejemplo, a Windows, o a otros servidores dentro de Linux, nos ceñiremos a este sistema primeramente por existir más documentación sobre él y también por la falta de fondos para pagar y mantener licencias.
\par Se nos cuenta, citando un tercer estudio, que Apache tiene limitaciones en el ancho de banda, PHP está limitado por la CPU y MySQL se ve enmarcado entre la capacidad de la memoria y el rendimiento de los discos duros. Por ello se recomienda que, al menos en la versión final, estos elementos sean compilados en la máquina de destino.
\subsubsection{Cuellos de botella}
\par Debemos analizar los cuellos de botella particulares a nuestra instalación (CPU, memoria, ancho de banda y rendimiento de entrada/salida). Para ello, se nos sugiere utilizar una serie de herramientas que podemos ver en el cuadro \ref{tab:perftools} y, en el caso de encontrar problemas significativos, es recomendable, entre otras cosas, pensar en los servicios adicionales que están disponibles en ese servidor y desactivar o trasladar los que sean posibles.

\begin{codigo}{Comandos útiles para buscar cuellos de botella}{perftools}
\begin{lstlisting}[language=bash,frame=single]
top
ps -aux
netstat -anp | sort -u
\end{lstlisting}
\end{codigo}

\subsubsection{Invitados indeseables}
\par Deberemos evitar invitados indeseables que consumen los recursos de nuestra máquina como por ejemplo: ``crawlers'',``spiders'', indexadores, agregadores, emisores de spam, ``malware'', etc.
Podemos paliar esta degradación añadiendo retrasos cuando se detecten esos agentes, vetando algunas partes del sitio web o remitiendo a las versiones almacenadas en caché.

\subsubsection{Rendimiento de Apache}
\par Como hemos dicho, hay que tener cuidado con las restricciones que se suelen imponer sobre el caudal asignado al servidor. Como en este caso, estará en nuestro poder, habrá que configurar con especial cuidado, entre otras, las opciones: MaxSpareServers, ServerLimit y MaxClients. Además, existen, según la versión, en el propio sitio de Apache así como en otros lugares de internet, guías para optimizar el rendimiento del servidor\cite{references:apachedocs}.

\subsubsection{Ajustes de PHP}
\par El punto débil, como ya he dicho, de PHP es su consumo de CPU. Para solucionar esto existen diversas optimizaciones en el mercado que habrá que analizar con atención: Alternate PHP Cache, Zend Optimizer\cite{references:drupalzend} o eAccelerator entre otras. Además, para profundizar en el tema, podríamos referirnos al libro ``Optimizing PHP''\cite{references:optimizingphp}.

\subsubsection{Ajustes de rendimiento de MySQL}
\par Para este caso, por lo extenso del tema, podemos remitirnos a un artículo de la propia Drupal\cite{references:tuningmysql}, que sin duda excede el alcance de este apartado.

\subsubsection{Ancho de banda}
\par Como ya he comentado, las imágenes y los vídeos pueden suponer una reducción drástica del rendimiento a la hora de ser servidos. Lo ideal es proporcionar acceso directo a ellos, eliminando, en la medida de lo posible, los pasos intermedios y los procesos ``al vuelo''.

\subsubsection{Consumo de recursos de Drupal}
\par Trataremos, en todo momento, de vigilar cómo se usa la memoria, qué módulos son los más lentos, qué consultas a la base de datos tardan más y cuáles de ellas serían innecesarias por duplicidad o por ser estáticas. Las soluciones pasarían por implantar módulos de caché de consultas y contenidos, desactivar, modificar o reemplazar los módulos más problemáticos y tratar de guardar una integridad y unicidad en las consultas a la base de datos.

\subsubsection{Configuración de Drupal hacia el rendimiento}
\par Obviamente, tras configurar todos los actores que influyen en el entorno de trabajo de Drupal (su pila de procesos) deberíamos optimizar la propia plataforma para evitar consumos innecesarios de recursos, como los producidos por tareas pesadas programadas (vía ``cron'') con mucha frecuencia, las páginas almacenables por su contenido estático y demás cuestiones que podemos gestionar desde el panel de administración de Drupal. Además, existen en el mercado optimizaciones específicas de Drupal orientadas a la mejor del rendimiento, es decir, distribuciones (recordemos que se trata de código libre), como por ejemplo Pressflow\cite{references:drupalpressflow}.

\subsection{Umbral de costes}
\par Una vez analizadas las necesidades del cliente en cuanto a rendimiento pasamos a los costes asumibles. Los costes asumibles en cuanto a hardware son relativamente bajos. Existen en el cliente, varios servidores dedicados a distintas tareas (ftp, ssh, http, mail) siendo la más importante, quizás, el correo electrónico. La topología de la red, por razones de seguridad, no me ha sido revelada en su totalidad, así que, digamos que debo pensar simplemente en las necesidades asumiendo que el coste asociado al servidor será relativamente bajo, llegando, en el máximo inicial, a una máquina dedicada. No descarto que en un futuro los costes vayan a aumentar debido a la afluencia de visitas, pero inicialmente, voy a contar, para este supuesto, con una máquina estándar que detallaré en la parte de requisitos de hardware.
De la parte de terminales móviles (handhelds) y estaciones físicas dentro del museo, serían modelos de perfil bajo y costes marginales, pudiéndose tratar de modelos no precisamente punteros. Todo esto desde la perspectiva de rendimiento. Estéticamente, intuyo que la empresa se inclinaría (es razonable) por una opción no demasiado agresiva o estridente con lo encontrado en el museo (pantallas lcd, líneas básicas, tecnología ``invisible'', etc.), lo que haría aumentar los costes moderadamente. De este particular no se han concretado números, aunque sí hay una disposición previa a proporcionar lo que sea necesario teniendo en cuenta las limitaciones presupuestarias (de fondos y de sometimiento a escrutinio exhaustivo) que hay en la administración pública.

\section{Requisitos de Hardware}
\par Teniendo todo lo anterior en cuenta, podemos listar una estimación de requisitos específicos de hardware.
\subsection{Desarrollo}
\par Actualmente, esta plataforma está siendo desarrollada en una estación de sobremesa relativamente modesta que contaría con unas características que yo considero medias (cuadro \ref{tab:workstation}). Las páginas cargan con relativa fluidez (aunque todavía no está cargada toda la base de datos) y eso teniendo en cuenta que corre sobre una máquina virtual como ya he descrito más arriba.

\begin{table}[h]
  \begin{center}
    \begin{itemize}
     \item Modelo: HP Compaq dc7700 Convertible Minitower
     \item Arquitectura: 32 bit
     \item Procesador: Intel(R) Core(TM)2 CPU 6600 @ 2.40GHz
     \item Memoria RAM: 2 x (DIMM 1GB 667MHz)
     \item Disco duro: SATA 232GiB (250GB) 
     \item Red: Ethernet Gigabit Network Connection (1Gbit/s)
    \end{itemize}
  \end{center}
  \caption{Características hardware de la plataforma de desarrollo}
  \label{tab:workstation}
\end{table}




\subsection{Servidor}

\par Para el servidor necesitaremos, en principio, cualquier máquina de potencia media actual (diciembre de 2010) o incluso, ligeramente inferior, pudiendo darse el caso de compartir, temporalmente hasta tener un conteo de los rendimientos, espacio con algún servidor ya utilizado.
Deberá contar, eso sí, con algunas características técnicas que se detallan en el cuadro \ref{tab:reqhardserv}. Cabe destacar que, para la instalación del núcleo de Drupal, bastaría con unos 3MiB de espacio de disco duro, pero si a todo eso le añadimos un sistema operativo, un servidor web, un sistema gestor de base de datos, una instalación de PHP y unos cuantos módulos, los requisitos pueden variar notablemente.

\begin{table}[h]
  \begin{center}
    \begin{itemize}
     \item Arquitectura: 32 o 64 bit
     \item Procesador: Doble núcleo @ >= 2,4GHz
     \item Memoria RAM: > 2GB (>= 667MHz)
     \item Disco duro: SATA >= 500GB (a ser posible, replicado en RAID)
     \item Red: Ethernet Gigabit
    \end{itemize}
  \end{center}
  \caption{Requisitos hardware del servidor}
  \label{tab:reqhardserv}
\end{table}


\subsection{Infraestructura de red}

\par Además, sería necesario disponer de la infraestructura adecuada de red. Si se decide que va a se este el que tenga la salida a Internet, deberá estar dotado del correspondiente \textbf{enlace} con un ``upstream'' aceptable para no verse asediado por el cuello de botella de la red general. Claro está, las medidas de seguridad pertinentes deberán ser instaladas (\textbf{firewall, segmentación, NIDS/NIPS}) y el \textbf{enrutamiento} necesario para obtener acceso tanto desde internet como desde la intranet de una supuesta wifi o una estación en sala (vía segmentación ethernet de la red).

\par Para desplegar la infraestructura de red inalámbrica en el recorrido de la visita, habrá que implementar \textbf{puntos de acceso} donde no los haya y cablear los mismos hasta la red principal. La cobertura de uno de los edificios (actualmente ya desplegada) deberá ser desdoblada para permitir una red de trabajo y otra de visitantes que sólo permita el acceso a los contenidos del museo virtual.

\subsection{Cliente en sala}

\par Para este cometido, valdrá cualquier equipo capaz de ejecutar un entorno gráfico, un navegador relativamente moderno y complementos de audio y vídeo no demasiado exigentes (cuadro \ref{ref:reqhardsala}). 

\begin{table}[h]
  \begin{center}
    \begin{itemize}
     \item Arquitectura: 32 bit
     \item Procesador: @ >= 1GHz
     \item Tarjeta de vídeo: >= 64MB RAM
     \item Memoria RAM: >= 1GB
     \item Disco duro: >= 100GB
     \item Red: Ethernet Gigabit
     \item Pantalla: >= 20 pulgadas, delgada, si es posible, táctil-capacitiva
    \end{itemize}
  \end{center}
  \caption{Requisitos hardware del cliente en sala}
  \label{ref:reqhardsala}
\end{table}


\subsection{Dispositivo móvil}

\par Aun tratándose de un dispositivo sin confirmar en esta fase del proyecto, creo conveniente agregarlo aquí. La idea es un dispositivo ergonómico, similar, en tamaño a una PDA y compatible con tecnologías parecidas a las del cliente en sala (cuadro \ref{ref:reqhardhheld}). Se discute en diferentes foros\cite{references:tatehandheld} por dónde, después de haber visto como la mayoría de los dispositivos, a menudo bajo contrato de leasing, utilizados en exposiciones se iban quedando obsoletos rápidamente, deberían ir los tiros a partir de ahora en cuestiones de visitas ``auto-guiadas''. Las conclusiones en materia de software están más o menos claras: se ha de crear una plataforma común (que bien podría ser un cms adaptado a plataformas móviles), incluso común a otras instituciones museísticas, encima de las cuales poder correr las ``aplicaciones'' necesarias para enriquecer y extender la experiencia del usuario en la exposición. En el lado hardware es donde se presenta la disyuntiva: seguir con el modelo de dispositivos bajo préstamo/alquiler, optar porque los propios visitantes accedan con sus teléfonos inteligentes al contenido que se ofrece, o una mezcla de estas dos. Bajo esa premisa, veo precipitado aventurar unas características, aunque ahí quedan, lo que yo consideraría, un dispositivo apto para esta función.

\begin{table}[h]
  \begin{center}
    \begin{itemize}
     \item Procesador: @ >= 500MHz
     \item Memoria RAM: >= 64MB
     \item Disco duro: >= 256MB
     \item Red: Wi-Fi 802.11g/n
     \item Pantalla: en color >= 4 pulgadas, a poder ser, táctil-capacitiva
     \item Batería: larga vida
     \item Audio: salida de audio
     \item Construcción: resistente, ligera y lo más simple posible en cuanto a piezas.
    \end{itemize}
  \end{center}
  \caption{Requisitos hardware del dispositivo móvil}
  \label{ref:reqhardhheld}
\end{table}


\section{Requisitos de Software}
\par Para el caso del software vamos a glosar lo que en principio tiene que ver con la opción elegida (Drupal).

\subsection{Desarrollo}
\par El software que corre encima de la máquina descrita en la sección anterior está, como ya expliqué contenido en una máquina virtual. Esta máquina virtual ha optado por un sistema \textbf{``LAMP''}, aunque es cierto que Drupal puede correr sobre Windows con IIS (más la pertinente implementación de PHP), sobre otros servidores (LightHTTPd) o sobre implementaciones específicas de PHP. Incluso, algunas distribuciones de Drupal permiten interactuar con otros sabores de sistema gestor de base de datos diferente de MySQL.

\subsection{Servidor}

\par El servidor, dado que sería, en lo posible, el trabajo resultado del desarrollo, debería estar bajo otro sistema LAMP. Visto lo visto, puede ser recomendable integrar un servidor con un sistema Debian Linux (por estabilidad) o \textbf{Ubuntu Server} (por similitud con la plataforma de desarrollo). Es cierto, eso sí, que, sea en una u otra máquina, debería llegarse a un acuerdo y, preparar, si el presupuesto lo permite, un ``servidor'' de pruebas, independiente del de desarrollo y del de producción. Se buscará que el software en ellos sea el mismo, dejando las versiones más estables para el servidor de producción, y siguiendo un ciclo clásico de \textbf{Do-Check-Act}\label{cha:docheckact} desde el entorno de desarrollo, al servidor de pruebas y después, si no hay errores, al de producción.

\par Así que, en principio, la estructura será la misma que para el entorno de desarrollo, aunque es previsible que la modificación inicial será al revés: una vez configurado un entorno estable para el servidor de producción, se igualarán el resto de servidores. A partir de ahí, el ciclo de actualizaciones y/o cambios seguirá el curso propuesto más arriba.

\subsection{Cliente en sala}

\par En este caso, los requisitos software son muy sencillos: un sistema operativo que soporte la pantalla táctil (en su caso) y un navegador moderno que sepa interpretar los estándares del \textbf{W3C} y reproducir contenido multimedia. Además, claro está, se deberán tomar las medidas de seguridad necesarias para que no haya brechas importantes de seguridad, ya sea a base de políticas, permisos, antivirus, etc.

\subsection{Dispositivo móvil}

\par Para el dispositivo móvil necesitaremos prácticamente lo mismo que hemos requerido para el cliente en sala. Se trata, en realidad de una réplica que trata de ofrecer los mismos contenidos ahorrando en transferencias y procesamiento, pero al fin y al cabo, lo mismo. Un sistema operativo optimizado para plataformas móviles (Android, Windows Phone, iPhoneOS, Symbian) que tenga conectividad wifi, incluya un navegador como los anteriormente descritos y permita, conscientes de las limitaciones de hardware, la reproducción de la mayoría de contenidos multimedia ofrecidos, si no todos.