\documentclass[spanish,11pt,final]{book}

% Paquetes
\usepackage[T1]{fontenc}        % Codificación de fuentes
\usepackage[utf8]{inputenc}     % Codificación para caracteres especiales
\usepackage{ae,aecompl}         % Usar fuentes vectoriales
\usepackage[spanish]{babel}     % Nombres en español
\usepackage{hyperref}           % Hiperenlaces en PDF
\usepackage{listings}		% Listados de código
\usepackage{cite}		% Citas
\usepackage{wrapfig}		% Escribir texo alrededor de imagenes
\usepackage{fancyhdr}		% Personalizar cabeceras y pies de pagina
\usepackage{makeidx}		% Crear indices
\usepackage{afterpage}		% Permite definir donde se situan las figuras a base de saltos de pagina
\usepackage{xspace}		% Evita que latex malinterprete macros
\usepackage{graphicx}		% Deja incluir imagenes


% \usepackage{epsfig}		% Para incluir figuras en eps
% \usepackage{verbatim}		% Incluir ficheros literalmente (listados de codigo}
% \usepackage{moreverb}		% Complementario a verbatim
% \usepackage{multicol}		% Intercalar bloques de una y varias columnas
% \usepackage{amsmath}		% Permite utilizar formulas matematicas
% \usepackage{eurosym}		% Simbolo del euro
% \usepackage{subfigure}	% Componer una figura de varias subfiguras
% \usepackage{multirow}		% Componer tablas multicolumna y multifila
% \usepackage{rotating}		% Permite incluir figuras con cierta rotacion
% \usepackage{supertabular}	% Extender las tablas a lo largo de varias paginas
% \usepackage{hhline}		% Manipular lineas horizontales
% \usepackage{array}		% Mas opciones a la hora de definir las filas
% \usepackage{longtable}	% Similar a supertabular


% Fuentes
%\usepackage{palatino}
%\usepackage{bookman}
%\usepackage{gfsartemisia}
\usepackage{utopia}





%se utiliza para numerar las \subsubsection
\setcounter{secnumdepth}{5}

%para hacer que las \subsubsection aparezcan en el indice
\setcounter{tocdepth}{5}

% Ruta de las imagenes
\graphicspath{{img/}}


% Definición de entornos nuevos

\newenvironment{narrow}[2]{%
\begin{list}{}{%
\setlength{\topsep}{0pt}%
\setlength{\leftmargin}{#1}%
\setlength{\rightmargin}{#2}%
\setlength{\listparindent}{\parindent}%
\setlength{\itemindent}{\parindent}%
\setlength{\parsep}{\parskip}}%
\item[]}{\end{list}}

\newenvironment{Lista}
{\begin{list}{}{\setlength{\leftmargin}{0pt}%
\setlength{\parsep}{0pt}%
\setlength{\rightmargin}{0pt}%
\setlength{\topsep}{2pt}}
\item \small{}
}
{\end{list}}

% Incluir código fuente
\newenvironment{codigo}[2]
{
\def\codecap{#1}%
\def\coderefer{#2}%
\begin{table}\begin{center}
}
{
  \end{center}
  \caption{\codecap}
  \label{tab:\coderefer}
\end{table}
}




% Definición de comandos nuevos
\newcommand{\eg}{e.g.,\xspace}
\newcommand{\ie}{i.e.,\xspace}
\newcommand{\etc}{etc.\@\xspace}
%\newcommand{\I}{\index}
\newcommand{\n}{\textbf}
\newcommand{\s}{\small}

\newcommand{\clearemptydoublepage}{\newpage{\pagestyle{empty}\cleardoublepage}}

% Figuras de comparativa
\newcommand{\fig}[2]{
  \begin{figure}[t!]
    \begin{narrow}{-1cm}{-1cm}
      \centering
      \includegraphics[width=\linewidth]{#1}
      \caption{#2}
      \label{fig:#1}
    \end{narrow}
  \end{figure}
}


% Redefinición de comandos
\renewcommand{\theenumi}{\arabic{enumi}}
\renewcommand{\theenumii}{\alph{enumii}}
\renewcommand{\theenumiii}{\roman{enumiii}}
\renewcommand{\tablename}{Tabla}
\renewcommand{\listtablename}{Índice de tablas}
\renewcommand{\c}{\emph}

\renewcommand{\chaptermark}[1]{\markboth{#1}{}}
\renewcommand{\sectionmark}[1]{\markright{\thesection. #1}}

%\voffset=-25pt
%\textheight=630pt
%\textwidth=430pt
\voffset=-2.7cm
%\hoffset=-0.5cm
\textheight=24cm
%\textwidth=17cm
\textwidth=16cm
\oddsidemargin=0pt
\evensidemargin=0pt
\pagestyle{fancy}

\lhead[\fancyplain{}{\bfseries\thepage}]{\fancyplain{}{\bfseries\rightmark}}
\rhead[\fancyplain{}{\bfseries\leftmark}]{\fancyplain{}{\bfseries\thepage}}
\cfoot{}

\makeatletter
\newcommand{\noapp}{\def\@chapapp{\chaptername}
  \renewcommand \thechapter{\arabic{chapter}}
}
\makeatother

%para generar el indice
\makeindex
